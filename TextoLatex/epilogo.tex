\chapter*{Epílogo}\label{chp:epilogo}
\addcontentsline{toc}{chapter}{Epílogo}

Este texto foi concebido para apoiar as aulas práticas da disciplina Redes de Comunicação I. Nem sempre é possível a utilização de equipamentos físicos em quantidade suficiente para os estudantes. Por isso, a utilização de um ambiente simulado, próximo do real, ajuda os estudantes a colocar em prática os conhecimentos adquiridos na disciplina.

Temos consciência que este texto não está completo. Portanto, as novas versões devem trazer mais conteúdo, além de corrigir eventuais erros da versão atual.  É importante destacar que o \CPT possui muitos recursos que não foram explorados aqui. Porém, a expectativa é que o conteúdo abordado nesse texto seja apenas o ponto de partida para outras práticas mais interessantes, que explorem situações mais particulares, e prepare os estudantes para os desafios no mundo real. 

Por fim, o leitor desse texto pode ficar à vontade para sugerir novas práticas ou novos conteúdos a serem abordados nas próximas versões desse texto. As críticas, sugestões e elogios podem ser feitas pelo e-mail a seguir ou nas redes sociais.

Bom proveito!

\href{mailto://gradvohl@ft.unicamp.br}{\faIcon{at}\xspace gradvohl@ft.unicamp.br}

\href{https://github.com/gradvohl}{\faIcon{github}/gradvohl}

\href{https://orcid.org/0000-0002-6520-9740}{\textcolor{orcidlogocol}{\faIcon{orcid}}/0000-0002-6520-9740}

\href{https://www.linkedin.com/in/andregradvohl}{\textcolor{linkedinlogocol}{\faIcon{linkedin}}/andregradvohl}

\href{https://twitter.com/AGradvohl}{\textcolor{twitterlogocol}{\faIcon{twitter}}/AGradvohl}

\href{https://fosstodon.org/@gradvohl}{\textcolor{mastodonlogocol}{\faIcon{mastodon}} @gradvohl@fosstodon.org}

\vspace*{\fill}
\begin{flushright}
    \textit{The greatest teacher, failure is.}\\
    Yoda.
\end{flushright}

